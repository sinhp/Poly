
In this chapter we provide several different perspectives on polynomial functors.

As a first introduction, polynomial functors are categorfication of the usual polynomials from algebra. Categorification is the process of finding category-theoretic analogs of set-theoretic concepts by replacing sets with categories, elements by objects, functions with functors, equations between elements by isomorphisms between objects, and equations between functions by natural isomorphisms between functors.

\renewcommand{\arraystretch}{1.5}

%% draw me a latex table here
\begin{table}[h]
  \centering
  \begin{tabular}{|c|c|}
    \hline
      \textbf{Sets/Types}  & \textbf{Categories} \\
      \hline
      $\Nat$ & The category $\FinSet$ of finite sets and functions \\
      \hline
      $2 \in \Nat$ & The set $\mathbf{2} = \{0, 1\}$\\
      \hline
      Equality $1 + 1 = 2$ &  Isomorphism $\mathbf{1} \coprod \mathbf{1} \iso \mathbf{2}$ in $\FinSet$ \\
      \hline
      Equality $2 \cdot 3 = 6$ &  Isomorphism $\mathbf{2} \times \mathbf{3} \iso \mathbf{6}$ in $\FinSet$ \\
      \hline
      Addition  operation
      &
      The disjoint union functor
      \\
      $+ : \mathtt{Nat} \times \mathtt{Nat} \to \mathtt{Nat}$ &
      $+ : \FinSet \times \FinSet \to \FinSet$ \\
      \hline
      Multiplication operation & The product functor \\
      $ \cdot : \mathtt{Nat} \times \mathtt{Nat} \to \mathtt{Nat}$
      &
       $\times : \FinSet \times \FinSet \to \FinSet$
      \\
      \hline
      Commutativity of addition  &     Natural isomorphism  \\
      $\forall (m, n : \Nat), m + n = n + m$ & $\mathbf{m} + \mathbf{n} \iso \mathbf{n} + \mathbf{m}$ in $\FinSet$ \\
      \hline
      Commutativity of multiplication  &     Natural isomorphism  \\
      $\forall (m, n : \Nat), m \cdot n = n \cdot m$ & $\mathbf{m} \times \mathbf{n} \iso \mathbf{n} \times \mathbf{m}$ in $\FinSet$ \\
      \hline
      Associativity of addition  &     Natural isomorphism  \\
      $\forall (m, n, k : \Nat), m + (n + k) = (m + n) + k$ & $\mathbf{m} + (\mathbf{n} + \mathbf{k}) \iso (\mathbf{m} + \mathbf{n}) + \mathbf{k}$ in $\FinSet$ \\
      \hline
      Associativity of multiplication  &     Natural isomorphism  \\
      $\forall (m, n, k : \Nat), m \cdot (n \cdot k) = (m \cdot n) \cdot k$ & $\mathbf{m} \times (\mathbf{n} \times \mathbf{k}) \iso (\mathbf{m} \times \mathbf{n}) \times \mathbf{k}$ in $\FinSet$ \\
      \hline
      $\vdots$ & $\vdots$ \\
      \hline
  \end{tabular}
\end{table}

In fact, $\Nat$ is a commutative semiring; it admits two monoid structure and can be seen as an additive commutative monoid $(\Nat, +, 0)$ and a multiplicative commutative monoid $(\Nat, \cdot, 1)$. The two operations are related by the distributive law $m \cdot (n + k) = m \cdot n + m \cdot k$. One category-level higher, $\FinSet$ can be seen as a commutative rig-category, that is a category with two monoidal structures, one for the product and one for the coproduct. The two monoidal structures are related by the distributive law $X \times (Y + Z) \iso (X \times Y) + (X \times Z)$. Iterating this process, we can climb the ladder of $n$-categories: The rig-categories are objects of a 2-category where 1-morphisms are monoidal functors and 2-morphisms are monoidal natural transformations.

Recall that a polynomial is a finite sum of monomials, where each monomial is a product of variables raised to some non-negative integer powers. A univariate polynomial is usually writted down as a formal sum
\begin{equation}
\label{eq:univariate-presentation-1}
\sum_{i=0}^{n - 1} \, a_i \, x^i
\end{equation}

where $x$ is the variable, $a_i$ are coefficients in some ring $R$. To have a unique representation, we sort the monomials by their exponent, the terms of lower exponent come first, so although $x + x^ 2$ and $x^2 + x$ are the same polynomial, we choose the first one as the canonical representation.

Another way to think about this formal sum is to see the exponent $i$ as a function of the coefficients: In this view, a a polynomial is simply a function $R \to \Nat$ with finite support. So we can write \ref{eq:univariate-presentation-1} as
\begin{equation}
\label{eq:univariate-presentation-2}
\sum_{c \in I} \, c \, x^{i(c)}
\end{equation}
where $I$ is a finite support. This is how usual polynomials are defined in the Lean4's mathematical library \texttt{mathlib}.

\begin{LeanCode}
  structure Polynomial (R : Type*) [Semiring R] where ofFinsupp ::
  toFinsupp : AddMonoidAlgebra R Nat
\end{LeanCode}

where \Lean{AddMonoidAlgebra R Nat} is defined as the type \Lean{R →₀ Nat} of functions from $R$ to $\Nat$ with finite support.

These two views are equivalent. For simplicity, let's restrict ourselves to the case of polynomials with coefficients in $\Nat$ so that when expanded all the coefficients become $1$ and instead we will have multiplicities of the monomials. So we can write the polynomial $x + 3 * x^2 $ as $x + x^2 + x^2 + x^2$. In this presentation we encode a polynomial by its degree (a natural number $n$) and a finite list of multiplicities $m_0, m_1, \ldots, m_{n - 1}$. This is just a function $m : \Nat \to \Nat$ with finite support.



A multivariate polynomial is a polynomial in several variables, say $x_1, \ldots, x_n$.






They are usually denoted by a sum of the form
$$\sum_{i \in I} \prod_{j \in J} a_{ij} x_j^{e_{ij}}$$
where $I$ and $J$ are finite sets, $a_{ij}$ are coefficients, and $e_{ij}$ are non-negative integers.
