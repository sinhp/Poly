In this section we develop some of the definitions
and lemmas related to polynomial endofunctors that we will use
in the rest of the notes.

\medskip

\begin{defn}[Polynomial endofunctor]
  \label{defn:UvPoly}
  \lean{CategoryTheory.UvPoly} \leanok
  Let $\catC$ be a locally Cartesian closed category
  (in our case, presheaves on the category of contexts).
  This means for each morphism $t : B \to A$ we have an adjoint triple
  % https://q.uiver.app/#q=WzAsMyxbMCwyXSxbMSwyLCJcXGNhdEMgLyBBIl0sWzEsMCwiXFxjYXRDIC8gQiJdLFsyLDEsInRfKiIsMCx7Im9mZnNldCI6LTV9XSxbMiwxLCJ0XyEiLDIseyJvZmZzZXQiOjV9XSxbMSwyLCJ0XioiLDFdLFs1LDMsIiIsMix7ImxldmVsIjoxLCJzdHlsZSI6eyJuYW1lIjoiYWRqdW5jdGlvbiJ9fV0sWzQsNSwiIiwyLHsibGV2ZWwiOjEsInN0eWxlIjp7Im5hbWUiOiJhZGp1bmN0aW9uIn19XV0=
  \begin{center}\begin{tikzcd}
    & {\catC / B} \\
    \\
    {} & {\catC / A}
    \arrow[""{name=0, anchor=center, inner sep=0}, "{t_*}", bend left, shift left=5, from=1-2, to=3-2]
    \arrow[""{name=1, anchor=center, inner sep=0}, "{t_!}"', bend right, shift right=5, from=1-2, to=3-2]
    \arrow[""{name=2, anchor=center, inner sep=0}, "{t^*}"{description}, from=3-2, to=1-2]
    \arrow["\dashv"{anchor=center}, draw=none, from=1, to=2]
    \arrow["\dashv"{anchor=center}, draw=none, from=2, to=0]
  \end{tikzcd}\end{center}
  where $t^{*}$ is pullback, and $t_{!}$ is composition with $t$.

  Let $t : B \to A$ be a morphism in $\catC$.
  Then define $\Poly{t} : \catC \to \catC$ be the composition
  \[
    \Poly{t} := A_{!} \circ t_{*} \circ B^{*}
    \quad \quad \quad
  \]
    % https://q.uiver.app/#q=WzAsNCxbMCwwLCJcXGNhdEMiXSxbMSwwLCJcXGNhdEMgLyBCIl0sWzIsMCwiXFxjYXRDIC8gQSJdLFszLDAsIlxcY2F0QyJdLFswLDEsIkJeKiJdLFsxLDIsInRfKiJdLFsyLDMsIkFfISJdXQ==
  \begin{center}\begin{tikzcd}
    \catC & {\catC / B} & {\catC / A} & \catC
    \arrow["{B^*}", from=1-1, to=1-2]
    \arrow["{t_*}", from=1-2, to=1-3]
    \arrow["{A_!}", from=1-3, to=1-4]
  \end{tikzcd}\end{center}
\end{defn}
