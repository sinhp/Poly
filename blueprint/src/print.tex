% This file makes a printable version of the blueprint
% It should include all the \usepackage needed for the pdf version.
% The template version assume you want to use a modern TeX compiler
% such as xeLaTeX or luaLaTeX including support for unicode
% and Latin Modern Math font with standard bugfixes applied.
% It also uses expl3 in order to support macros related to the dependency graph.
% It also includes standard AMS packages (and their improved version
% mathtools) as well as support for links with a sober decoration
% (no ugly rectangles around links).
% It is otherwise a very minimal preamble (you should probably at least
% add cleveref and tikz-cd).

\documentclass[a4paper]{report}

\usepackage{geometry}

\usepackage{expl3}

\usepackage{amssymb, amsthm, mathtools}
\usepackage[unicode,colorlinks=true,linkcolor=blue,urlcolor=magenta, citecolor=blue]{hyperref}

\usepackage[warnings-off={mathtools-colon,mathtools-overbracket}]{unicode-math}

\usepackage{amsmath,stmaryrd,enumerate,a4,array}
\usepackage{geometry}
\usepackage{url}
\usepackage[cmtip,all]{xy}
\usepackage{subcaption}
\usepackage{cleveref}
\numberwithin{equation}{section}
\usepackage[parfill]{parskip}
\usepackage{tikz, tikz-cd, float} % Commutative Diagrams
% \usepackage{bussproofs}
\usepackage{graphics}

\usepackage{fontspec}
\usepackage{minted}
% \usepackage{verbatim}

% autogobble lets us indent the lean code to distinguish it from the latex
\setminted{frame=single}
\newmintinline[Lean]{lean}{breaklines,breakbefore={. },breakafter={_}}
\newminted[LeanCode]{Lean}{autogobble}


% In this file you should put all LaTeX macros and settings to be used both by
% the pdf version and the web version.
% This should be most of your macros.

% The theorem-like environments defined below are those that appear by default
% in the dependency graph. See the README of leanblueprint if you need help to
% customize this.
% The configuration below use the theorem counter for all those environments
% (this is what the [theorem] arguments mean) and never resets it.
% If you want for instance to number them within chapters then you can add
% [chapter] at the end of the next line.

%%%%%%% THEOREMS %%%%%%%%%
\newtheorem{theorem}{Theorem}[section]
\newtheorem*{theorem*}{Theorem}
\newtheorem{prop}[theorem]{Proposition}
\newtheorem{obs}[theorem]{Observation}
\newtheorem{lemma}[theorem]{Lemma}
\newtheorem{cor}[theorem]{Corollary}
\newtheorem{applemma}{Lemma}[section]

\theoremstyle{definition}
\newtheorem{definition}[theorem]{Definition}

\theoremstyle{remark}
\newtheorem{rmk}[theorem]{Remark}
\newtheorem*{eg}{Example}
\newtheorem{ex}{Exercise}
\newtheorem*{remark*}{Remark}
\newtheorem*{remarks*}{Remarks}
\newtheorem*{notation*}{Notation}
\newtheorem*{convention*}{Convention}

% \newenvironment{comment}{\begin{quote} \em Comment. }{\end{quote}}

%%%%%%% Relations

\newcommand{\defeq}{=_{\mathrm{def}}}
\newcommand{\iso}{\cong}
\newcommand{\equi}{\simeq}
\newcommand{\retequi}{\unlhd}
\newcommand{\op}{\mathrm{op}}
\newcommand{\Nat}{\mathbf{N}}
% \newcommand{\co}{\colon}
\newcommand{\st}{\,|\,}


%%% Morphisms

\newcommand{\id}{\mathsf{id}}
\newcommand{\yo}{\mathsf{y}}
\newcommand{\Arr}{\mathsf{Arr}}
\newcommand{\CartArr}{\mathsf{CartArr}}
\newcommand{\unit}{\, \mathsf{unit} \, }
\newcommand{\counit}{\, \mathsf{counit} \, }

%%%% Objects

\newcommand{\tcat}{\mathbf}
\newcommand{\catC}{\mathbb{C}}
\newcommand{\pshC}{\psh{\catC}}
\newcommand{\psh}[1]{\mathbf{Psh}(#1)} %% consider renaming to \Psh
\newcommand{\PSH}[1]{\mathbf{PSH}(#1)}
\newcommand{\set}{\tcat{set}}
\newcommand{\Set}{\tcat{Set}}
\newcommand{\FinSet}{\tcat{Set}_\mathrm{fin}}
\newcommand{\SET}{\tcat{SET}}
\newcommand{\cat}{\tcat{cat}}
\newcommand{\ptcat}{\tcat{cat}_\bullet}
\newcommand{\Cat}{\tcat{Cat}}
\newcommand{\ptCat}{\tcat{Cat}_\bullet}
\newcommand{\grpd}{\tcat{grpd}}
\newcommand{\Grpd}{\tcat{Grpd}}
\newcommand{\ptgrpd}{\tcat{grpd}_\bullet}
\newcommand{\ptGrpd}{\tcat{Grpd}_\bullet}
\newcommand{\Pshgrpd}{\mathbf{Psh}(\grpd)}
\newcommand{\PshCat}{\mathbf{Psh}(\Cat)}

\newcommand{\terminal}{\bullet}
\newcommand{\Two}{\bullet+\bullet}

%%% Polynomials
\newcommand{\polyspan}[7]{#1 \xleftarrow{#5} #2 \xrightarrow{#6} #3 \xrightarrow{#7} #4}
\newcommand{\PolyComp}{\lhd}
\newcommand{\upP}{\mathrm{P}}
\newcommand{\Poly}[1]{P_{#1}}
\newcommand{\Star}[1]{{#1}^{\star}}


%%%% Syntax

\newcommand{\Type}{\mathsf{Ty}}
\newcommand{\Term}{\mathsf{Tm}}
\newcommand{\tp}{\mathsf{tp}}
\newcommand{\disp}[1]{\mathsf{disp}_{#1}}
\newcommand{\var}{\mathsf{var}}
\newcommand{\Prop}{\mathsf{Prop}}
\newcommand{\U}{\mathsf{U}}
\newcommand{\E}{\mathsf{E}}
\newcommand{\El}{\mathsf{El}}
\newcommand{\pair}{\mathsf{pair}}
\newcommand{\Id}{\mathsf{Id}}
\newcommand{\refl}{\mathsf{refl}}
\newcommand{\J}{\mathsf{J}}
\newcommand{\fst}{\mathsf{fst}}
\newcommand{\snd}{\mathsf{snd}}
\newcommand{\ev}[2]{\mathsf{ev}_{#1} \, #2}
\newcommand{\dom}{\mathsf{dom}}
\newcommand{\cod}{\mathsf{cod}}
\newcommand{\Exp}{\mathsf{Exp}}
\newcommand{\fun}{\mathsf{fun}}
\newcommand{\name}[1]{\ulcorner #1 \urcorner}

\newcommand{\Fib}{\mathsf{Fib}}
\newcommand{\lift}[2]{\mathsf{lift} \, #1 \, #2}
\newcommand{\fiber}{\mathsf{fiber}}
\newcommand{\Interval}{\mathbb{I}}
\newcommand{\Lift}[2]{\mathsf{L}_{#1}^{#2}}

%%%% Interpretation

\newcommand{\doublesquarelbracket}{[\![}
\newcommand{\doublesquarerbracket}{]\!]}
\newcommand{\IntpCtx}[1]{\doublesquarelbracket #1 \doublesquarerbracket}
\newcommand{\IntpType}[1]{\doublesquarelbracket #1 \doublesquarerbracket}
\newcommand{\IntpTerm}[1]{\doublesquarelbracket #1 \doublesquarerbracket}

% % Greek
\newcommand{\al}{\alpha}
\newcommand{\be}{\beta}
\newcommand{\ga}{\gamma}
\newcommand{\de}{\delta}
\newcommand{\ep}{\varepsilon}
\newcommand{\io}{\iota}
\newcommand{\ka}{\kappa}
\newcommand{\la}{\lambda}
\newcommand{\om}{\omega}
\newcommand{\si}{\sigma}

\newcommand{\Ga}{\Gamma}
\newcommand{\De}{\Delta}
\newcommand{\Th}{\Theta}
\newcommand{\La}{\Lambda}
\newcommand{\Si}{\Sigma}
\newcommand{\Om}{\Omega}

% % Families
\newcommand{\setbn}[2]{\left\{\left. #1 \ \middle\rvert\ #2 \right.\right\}}
\newcommand{\seqbn}[2]{\left(\left. #1 \ \middle\rvert\ #2 \right.\right)}
\newcommand{\sqseqbn}[2]{\left[\left. #1 \ \middle\rvert\ #2 \right.\right]}
\newcommand{\append}{\mathbin{^\frown}}


% % Misc
% \newenvironment{bprooftree}
%   {\leavevmode\hbox\bgroup}
%   {\DisplayProof\egroup}

% This file makes a printable version of the blueprint
% It should include all the \usepackage needed for the pdf version.
% The template version assume you want to use a modern TeX compiler
% such as xeLaTeX or luaLaTeX including support for unicode
% and Latin Modern Math font with standard bugfixes applied.
% It also uses expl3 in order to support macros related to the dependency graph.
% It also includes standard AMS packages (and their improved version
% mathtools) as well as support for links with a sober decoration
% (no ugly rectangles around links).
% It is otherwise a very minimal preamble (you should probably at least
% add cleveref and tikz-cd).

\documentclass[a4paper]{report}

\usepackage{geometry}

\usepackage{expl3}

\usepackage{amssymb, amsthm, mathtools}
\usepackage[unicode,colorlinks=true,linkcolor=blue,urlcolor=magenta, citecolor=blue]{hyperref}

\usepackage[warnings-off={mathtools-colon,mathtools-overbracket}]{unicode-math}

\usepackage{amsmath,stmaryrd,enumerate,a4,array}
\usepackage{geometry}
\usepackage{url}
\usepackage[cmtip,all]{xy}
\usepackage{subcaption}
\usepackage{cleveref}
\numberwithin{equation}{section}
\usepackage[parfill]{parskip}
\usepackage{tikz, tikz-cd, float} % Commutative Diagrams
% \usepackage{bussproofs}
\usepackage{graphics}

\usepackage{fontspec}
\usepackage{minted}
\usepackage{verbatim}

% autogobble lets us indent the lean code to distinguish it from the latex
\setminted{frame=single}
\newmintinline[Lean]{lean}{breaklines,breakbefore={. },breakafter={_}}
\newminted[LeanCode]{Lean}{autogobble}


% In this file you should put all LaTeX macros and settings to be used both by
% the pdf version and the web version.
% This should be most of your macros.

% The theorem-like environments defined below are those that appear by default
% in the dependency graph. See the README of leanblueprint if you need help to
% customize this.
% The configuration below use the theorem counter for all those environments
% (this is what the [theorem] arguments mean) and never resets it.
% If you want for instance to number them within chapters then you can add
% [chapter] at the end of the next line.

%%%%%%% THEOREMS %%%%%%%%%
\newtheorem{theorem}{Theorem}[section]
\newtheorem*{theorem*}{Theorem}
\newtheorem{prop}[theorem]{Proposition}
\newtheorem{obs}[theorem]{Observation}
\newtheorem{lemma}[theorem]{Lemma}
\newtheorem{cor}[theorem]{Corollary}
\newtheorem{applemma}{Lemma}[section]

\theoremstyle{definition}
\newtheorem{definition}[theorem]{Definition}

\theoremstyle{remark}
\newtheorem{rmk}[theorem]{Remark}
\newtheorem*{eg}{Example}
\newtheorem{ex}{Exercise}
\newtheorem*{remark*}{Remark}
\newtheorem*{remarks*}{Remarks}
\newtheorem*{notation*}{Notation}
\newtheorem*{convention*}{Convention}

% \newenvironment{comment}{\begin{quote} \em Comment. }{\end{quote}}

%%%%%%% Relations

\newcommand{\defeq}{=_{\mathrm{def}}}
\newcommand{\iso}{\cong}
\newcommand{\equi}{\simeq}
\newcommand{\retequi}{\unlhd}
\newcommand{\op}{\mathrm{op}}
\newcommand{\Nat}{\mathbf{N}}
% \newcommand{\co}{\colon}
\newcommand{\st}{\,|\,}


%%% Morphisms

\newcommand{\id}{\mathsf{id}}
\newcommand{\yo}{\mathsf{y}}
\newcommand{\Arr}{\mathsf{Arr}}
\newcommand{\CartArr}{\mathsf{CartArr}}
\newcommand{\unit}{\, \mathsf{unit} \, }
\newcommand{\counit}{\, \mathsf{counit} \, }

%%%% Objects

\newcommand{\tcat}{\mathbf}
\newcommand{\catC}{\mathbb{C}}
\newcommand{\pshC}{\psh{\catC}}
\newcommand{\psh}[1]{\mathbf{Psh}(#1)} %% consider renaming to \Psh
\newcommand{\PSH}[1]{\mathbf{PSH}(#1)}
\newcommand{\set}{\tcat{set}}
\newcommand{\Set}{\tcat{Set}}
\newcommand{\FinSet}{\tcat{Set}_\mathrm{fin}}
\newcommand{\SET}{\tcat{SET}}
\newcommand{\cat}{\tcat{cat}}
\newcommand{\ptcat}{\tcat{cat}_\bullet}
\newcommand{\Cat}{\tcat{Cat}}
\newcommand{\ptCat}{\tcat{Cat}_\bullet}
\newcommand{\grpd}{\tcat{grpd}}
\newcommand{\Grpd}{\tcat{Grpd}}
\newcommand{\ptgrpd}{\tcat{grpd}_\bullet}
\newcommand{\ptGrpd}{\tcat{Grpd}_\bullet}
\newcommand{\Pshgrpd}{\mathbf{Psh}(\grpd)}
\newcommand{\PshCat}{\mathbf{Psh}(\Cat)}

\newcommand{\terminal}{\bullet}
\newcommand{\Two}{\bullet+\bullet}

%%% Polynomials
\newcommand{\polyspan}[7]{#1 \xleftarrow{#5} #2 \xrightarrow{#6} #3 \xrightarrow{#7} #4}
\newcommand{\PolyComp}{\lhd}
\newcommand{\upP}{\mathrm{P}}
\newcommand{\Poly}[1]{P_{#1}}
\newcommand{\Star}[1]{{#1}^{\star}}


%%%% Syntax

\newcommand{\Type}{\mathsf{Ty}}
\newcommand{\Term}{\mathsf{Tm}}
\newcommand{\tp}{\mathsf{tp}}
\newcommand{\disp}[1]{\mathsf{disp}_{#1}}
\newcommand{\var}{\mathsf{var}}
\newcommand{\Prop}{\mathsf{Prop}}
\newcommand{\U}{\mathsf{U}}
\newcommand{\E}{\mathsf{E}}
\newcommand{\El}{\mathsf{El}}
\newcommand{\pair}{\mathsf{pair}}
\newcommand{\Id}{\mathsf{Id}}
\newcommand{\refl}{\mathsf{refl}}
\newcommand{\J}{\mathsf{J}}
\newcommand{\fst}{\mathsf{fst}}
\newcommand{\snd}{\mathsf{snd}}
\newcommand{\ev}[2]{\mathsf{ev}_{#1} \, #2}
\newcommand{\dom}{\mathsf{dom}}
\newcommand{\cod}{\mathsf{cod}}
\newcommand{\Exp}{\mathsf{Exp}}
\newcommand{\fun}{\mathsf{fun}}
\newcommand{\name}[1]{\ulcorner #1 \urcorner}

\newcommand{\Fib}{\mathsf{Fib}}
\newcommand{\lift}[2]{\mathsf{lift} \, #1 \, #2}
\newcommand{\fiber}{\mathsf{fiber}}
\newcommand{\Interval}{\mathbb{I}}
\newcommand{\Lift}[2]{\mathsf{L}_{#1}^{#2}}

%%%% Interpretation

\newcommand{\doublesquarelbracket}{[\![}
\newcommand{\doublesquarerbracket}{]\!]}
\newcommand{\IntpCtx}[1]{\doublesquarelbracket #1 \doublesquarerbracket}
\newcommand{\IntpType}[1]{\doublesquarelbracket #1 \doublesquarerbracket}
\newcommand{\IntpTerm}[1]{\doublesquarelbracket #1 \doublesquarerbracket}

% % Greek
\newcommand{\al}{\alpha}
\newcommand{\be}{\beta}
\newcommand{\ga}{\gamma}
\newcommand{\de}{\delta}
\newcommand{\ep}{\varepsilon}
\newcommand{\io}{\iota}
\newcommand{\ka}{\kappa}
\newcommand{\la}{\lambda}
\newcommand{\om}{\omega}
\newcommand{\si}{\sigma}

\newcommand{\Ga}{\Gamma}
\newcommand{\De}{\Delta}
\newcommand{\Th}{\Theta}
\newcommand{\La}{\Lambda}
\newcommand{\Si}{\Sigma}
\newcommand{\Om}{\Omega}

% % Families
\newcommand{\setbn}[2]{\left\{\left. #1 \ \middle\rvert\ #2 \right.\right\}}
\newcommand{\seqbn}[2]{\left(\left. #1 \ \middle\rvert\ #2 \right.\right)}
\newcommand{\sqseqbn}[2]{\left[\left. #1 \ \middle\rvert\ #2 \right.\right]}
\newcommand{\append}{\mathbin{^\frown}}


% % Misc
% \newenvironment{bprooftree}
%   {\leavevmode\hbox\bgroup}
%   {\DisplayProof\egroup}

% This file makes a printable version of the blueprint
% It should include all the \usepackage needed for the pdf version.
% The template version assume you want to use a modern TeX compiler
% such as xeLaTeX or luaLaTeX including support for unicode
% and Latin Modern Math font with standard bugfixes applied.
% It also uses expl3 in order to support macros related to the dependency graph.
% It also includes standard AMS packages (and their improved version
% mathtools) as well as support for links with a sober decoration
% (no ugly rectangles around links).
% It is otherwise a very minimal preamble (you should probably at least
% add cleveref and tikz-cd).

\documentclass[a4paper]{report}

\usepackage{geometry}

\usepackage{expl3}

\usepackage{amssymb, amsthm, mathtools}
\usepackage[unicode,colorlinks=true,linkcolor=blue,urlcolor=magenta, citecolor=blue]{hyperref}

\usepackage[warnings-off={mathtools-colon,mathtools-overbracket}]{unicode-math}

\usepackage{amsmath,stmaryrd,enumerate,a4,array}
\usepackage{geometry}
\usepackage{url}
\usepackage[cmtip,all]{xy}
\usepackage{subcaption}
\usepackage{cleveref}
\numberwithin{equation}{section}
\usepackage[parfill]{parskip}
\usepackage{tikz, tikz-cd, float} % Commutative Diagrams
% \usepackage{bussproofs}
\usepackage{graphics}

\usepackage{fontspec}
\usepackage{minted}
\usepackage{verbatim}

% autogobble lets us indent the lean code to distinguish it from the latex
\setminted{frame=single}
\newmintinline[Lean]{lean}{breaklines,breakbefore={. },breakafter={_}}
\newminted[LeanCode]{Lean}{autogobble}


% In this file you should put all LaTeX macros and settings to be used both by
% the pdf version and the web version.
% This should be most of your macros.

% The theorem-like environments defined below are those that appear by default
% in the dependency graph. See the README of leanblueprint if you need help to
% customize this.
% The configuration below use the theorem counter for all those environments
% (this is what the [theorem] arguments mean) and never resets it.
% If you want for instance to number them within chapters then you can add
% [chapter] at the end of the next line.

%%%%%%% THEOREMS %%%%%%%%%
\newtheorem{theorem}{Theorem}[section]
\newtheorem*{theorem*}{Theorem}
\newtheorem{prop}[theorem]{Proposition}
\newtheorem{obs}[theorem]{Observation}
\newtheorem{lemma}[theorem]{Lemma}
\newtheorem{cor}[theorem]{Corollary}
\newtheorem{applemma}{Lemma}[section]

\theoremstyle{definition}
\newtheorem{definition}[theorem]{Definition}

\theoremstyle{remark}
\newtheorem{rmk}[theorem]{Remark}
\newtheorem*{eg}{Example}
\newtheorem{ex}{Exercise}
\newtheorem*{remark*}{Remark}
\newtheorem*{remarks*}{Remarks}
\newtheorem*{notation*}{Notation}
\newtheorem*{convention*}{Convention}

% \newenvironment{comment}{\begin{quote} \em Comment. }{\end{quote}}

%%%%%%% Relations

\newcommand{\defeq}{=_{\mathrm{def}}}
\newcommand{\iso}{\cong}
\newcommand{\equi}{\simeq}
\newcommand{\retequi}{\unlhd}
\newcommand{\op}{\mathrm{op}}
\newcommand{\Nat}{\mathbf{N}}
% \newcommand{\co}{\colon}
\newcommand{\st}{\,|\,}


%%% Morphisms

\newcommand{\id}{\mathsf{id}}
\newcommand{\yo}{\mathsf{y}}
\newcommand{\Arr}{\mathsf{Arr}}
\newcommand{\CartArr}{\mathsf{CartArr}}
\newcommand{\unit}{\, \mathsf{unit} \, }
\newcommand{\counit}{\, \mathsf{counit} \, }

%%%% Objects

\newcommand{\tcat}{\mathbf}
\newcommand{\catC}{\mathbb{C}}
\newcommand{\pshC}{\psh{\catC}}
\newcommand{\psh}[1]{\mathbf{Psh}(#1)} %% consider renaming to \Psh
\newcommand{\PSH}[1]{\mathbf{PSH}(#1)}
\newcommand{\set}{\tcat{set}}
\newcommand{\Set}{\tcat{Set}}
\newcommand{\FinSet}{\tcat{Set}_\mathrm{fin}}
\newcommand{\SET}{\tcat{SET}}
\newcommand{\cat}{\tcat{cat}}
\newcommand{\ptcat}{\tcat{cat}_\bullet}
\newcommand{\Cat}{\tcat{Cat}}
\newcommand{\ptCat}{\tcat{Cat}_\bullet}
\newcommand{\grpd}{\tcat{grpd}}
\newcommand{\Grpd}{\tcat{Grpd}}
\newcommand{\ptgrpd}{\tcat{grpd}_\bullet}
\newcommand{\ptGrpd}{\tcat{Grpd}_\bullet}
\newcommand{\Pshgrpd}{\mathbf{Psh}(\grpd)}
\newcommand{\PshCat}{\mathbf{Psh}(\Cat)}

\newcommand{\terminal}{\bullet}
\newcommand{\Two}{\bullet+\bullet}

%%% Polynomials
\newcommand{\polyspan}[7]{#1 \xleftarrow{#5} #2 \xrightarrow{#6} #3 \xrightarrow{#7} #4}
\newcommand{\PolyComp}{\lhd}
\newcommand{\upP}{\mathrm{P}}
\newcommand{\Poly}[1]{P_{#1}}
\newcommand{\Star}[1]{{#1}^{\star}}


%%%% Syntax

\newcommand{\Type}{\mathsf{Ty}}
\newcommand{\Term}{\mathsf{Tm}}
\newcommand{\tp}{\mathsf{tp}}
\newcommand{\disp}[1]{\mathsf{disp}_{#1}}
\newcommand{\var}{\mathsf{var}}
\newcommand{\Prop}{\mathsf{Prop}}
\newcommand{\U}{\mathsf{U}}
\newcommand{\E}{\mathsf{E}}
\newcommand{\El}{\mathsf{El}}
\newcommand{\pair}{\mathsf{pair}}
\newcommand{\Id}{\mathsf{Id}}
\newcommand{\refl}{\mathsf{refl}}
\newcommand{\J}{\mathsf{J}}
\newcommand{\fst}{\mathsf{fst}}
\newcommand{\snd}{\mathsf{snd}}
\newcommand{\ev}[2]{\mathsf{ev}_{#1} \, #2}
\newcommand{\dom}{\mathsf{dom}}
\newcommand{\cod}{\mathsf{cod}}
\newcommand{\Exp}{\mathsf{Exp}}
\newcommand{\fun}{\mathsf{fun}}
\newcommand{\name}[1]{\ulcorner #1 \urcorner}

\newcommand{\Fib}{\mathsf{Fib}}
\newcommand{\lift}[2]{\mathsf{lift} \, #1 \, #2}
\newcommand{\fiber}{\mathsf{fiber}}
\newcommand{\Interval}{\mathbb{I}}
\newcommand{\Lift}[2]{\mathsf{L}_{#1}^{#2}}

%%%% Interpretation

\newcommand{\doublesquarelbracket}{[\![}
\newcommand{\doublesquarerbracket}{]\!]}
\newcommand{\IntpCtx}[1]{\doublesquarelbracket #1 \doublesquarerbracket}
\newcommand{\IntpType}[1]{\doublesquarelbracket #1 \doublesquarerbracket}
\newcommand{\IntpTerm}[1]{\doublesquarelbracket #1 \doublesquarerbracket}

% % Greek
\newcommand{\al}{\alpha}
\newcommand{\be}{\beta}
\newcommand{\ga}{\gamma}
\newcommand{\de}{\delta}
\newcommand{\ep}{\varepsilon}
\newcommand{\io}{\iota}
\newcommand{\ka}{\kappa}
\newcommand{\la}{\lambda}
\newcommand{\om}{\omega}
\newcommand{\si}{\sigma}

\newcommand{\Ga}{\Gamma}
\newcommand{\De}{\Delta}
\newcommand{\Th}{\Theta}
\newcommand{\La}{\Lambda}
\newcommand{\Si}{\Sigma}
\newcommand{\Om}{\Omega}

% % Families
\newcommand{\setbn}[2]{\left\{\left. #1 \ \middle\rvert\ #2 \right.\right\}}
\newcommand{\seqbn}[2]{\left(\left. #1 \ \middle\rvert\ #2 \right.\right)}
\newcommand{\sqseqbn}[2]{\left[\left. #1 \ \middle\rvert\ #2 \right.\right]}
\newcommand{\append}{\mathbin{^\frown}}


% % Misc
% \newenvironment{bprooftree}
%   {\leavevmode\hbox\bgroup}
%   {\DisplayProof\egroup}

% This file makes a printable version of the blueprint
% It should include all the \usepackage needed for the pdf version.
% The template version assume you want to use a modern TeX compiler
% such as xeLaTeX or luaLaTeX including support for unicode
% and Latin Modern Math font with standard bugfixes applied.
% It also uses expl3 in order to support macros related to the dependency graph.
% It also includes standard AMS packages (and their improved version
% mathtools) as well as support for links with a sober decoration
% (no ugly rectangles around links).
% It is otherwise a very minimal preamble (you should probably at least
% add cleveref and tikz-cd).

\documentclass[a4paper]{report}

\usepackage{geometry}

\usepackage{expl3}

\usepackage{amssymb, amsthm, mathtools}
\usepackage[unicode,colorlinks=true,linkcolor=blue,urlcolor=magenta, citecolor=blue]{hyperref}

\usepackage[warnings-off={mathtools-colon,mathtools-overbracket}]{unicode-math}

\usepackage{amsmath,stmaryrd,enumerate,a4,array}
\usepackage{geometry}
\usepackage{url}
\usepackage[cmtip,all]{xy}
\usepackage{subcaption}
\usepackage{cleveref}
\numberwithin{equation}{section}
\usepackage[parfill]{parskip}
\usepackage{tikz, tikz-cd, float} % Commutative Diagrams
% \usepackage{bussproofs}
\usepackage{graphics}

\usepackage{fontspec}
\usepackage{minted}
\usepackage{verbatim}

% autogobble lets us indent the lean code to distinguish it from the latex
\setminted{frame=single}
\newmintinline[Lean]{lean}{breaklines,breakbefore={. },breakafter={_}}
\newminted[LeanCode]{Lean}{autogobble}


\input{macros/common}
\input{macros/print}

\title{Polynomial Functors in Lean 4}
\author{Sina Hazratpour}

\input{chapter/all}


\title{Polynomial Functors in Lean 4}
\author{Sina Hazratpour}

% This file collects all the blueprint chapters.
% It is not supposed to be built standalone:
% see web.tex and print.tex instead.

\title{Polynomial Functors in Lean4}
\author{Sina Hazratpour}

\begin{document}
\maketitle

\chapter{Locally Cartesian Closed Categories}
\input{chapter/lccc}
\chapter{Univaiate Polynomial Functors}
\input{chapter/uvpoly}
\chapter{Multivariate Polynomial Functors}
\input{chapter/mvpoly}
\chapter{Test}
\input{chapter/mvpoly}

% \bibliography{refs.bib}{}
% \bibliographystyle{alpha}

\end{document}



\title{Polynomial Functors in Lean 4}
\author{Sina Hazratpour}

% This file collects all the blueprint chapters.
% It is not supposed to be built standalone:
% see web.tex and print.tex instead.

\title{Polynomial Functors in Lean4}
\author{Sina Hazratpour}

\begin{document}
\maketitle

\chapter{Locally Cartesian Closed Categories}
\begin{definition}[exponentiable morphism]
  \label{defn:exponentiable-morphism}
  \lean{CategoryTheory.ExponentiableMorphism}
  \leanok
    Suppose $\catC$ is a category with pullbacks.
    A morphism $f \colon A \to B$ in $\catC$ is \textbf{exponentiable} if the pullback functor
    $f^* \colon \catC / B \to \catC / A$ has a right adjoint $f_*$.
    Since $f^*$ always has a left adjoint $f_!$, given by post-composition with $f$, an
    exponentiable morphism $f$ gives rise to an adjoint triple
    \begin{center}\begin{tikzcd}
      & {\catC / B} \\
      \\
      {} & {\catC / A}
      \arrow[""{name=0, anchor=center, inner sep=0}, "{f_*}", bend left, shift left=5, from=1-2, to=3-2]
      \arrow[""{name=1, anchor=center, inner sep=0}, "{f_!}"', bend right, shift right=5, from=1-2, to=3-2]
      \arrow[""{name=2, anchor=center, inner sep=0}, "{f^*}"{description}, from=3-2, to=1-2]
      \arrow["\dashv"{anchor=center}, draw=none, from=1, to=2]
      \arrow["\dashv"{anchor=center}, draw=none, from=2, to=0]
    \end{tikzcd}\end{center}
\end{definition}

\begin{definition}[pushforward functor]
  \label{rmk:pushforward}
  \lean{CategoryTheory.ExponentiableMorphism.pushforward}
  \leanok
  Let $f \colon A \to B$ be an exponentiable morphism in a category $\catC$ with pullbacks.
  We call the right adjoint $f_*$ of the pullback functor $f^*$ the \textbf{pushforward} functor
  along $f$.
\end{definition}

\begin{theorem}[exponentiable morphisms are exponentiable objects of the slices]\label{thm:exponentiable-mor-exponentiable-obj}
  \lean{CategoryTheory.ExponentiableMorphism.OverMkHom}
  \leanok
  \uses{defn:exponentiable-morphism}
  A morphism $f \colon A \to B$ in a category $\catC$ with pullbacks is exponentiable if and only if
  it is an exponentiable object, regarded as an object of the slice $\catC / B$.
\end{theorem}


\begin{definition}[Locally cartesian closed categories]\label{defn:LCCC}
\lean{CategoryTheory.LocallyCartesianClosed} \leanok
  A category with pullbacks is \textbf{locally cartesian closed} if
  is a category $\catC$ with a terminal object $1$ and with all slices $\catC / A$ cartesian closed.
\end{definition}

\chapter{Univaiate Polynomial Functors}
\input{chapter/uvpoly}
\chapter{Multivariate Polynomial Functors}
<<<<<<< HEAD
\begin{def}
  We define a polynomial over C to be a diagram $F$ in C of shape
  \begin{equation}
  \lean{MvPoly}
  \label{equ:poly}
  \xymatrix{
  I  & B \ar[l]_s \ar[r]^f & A \ar[r]^t & J \, . }
  \end{equation}
  We define $\poly{F}:\slice I \to \slice J$ as the composite
  \[
  \xymatrix{
  \slice{I} \ar[r]^{\pbk{s}} & \slice{B} \ar[r]^{\radj{f}} & \slice{A} \ar[r]^{\ladj{t}} & \slice{J} \, . }
  \]
  We refer to $\poly{F}$ as the polynomial functor associated to $F$, or
  the extension of $F$, and say that $F$ represents $\poly F$.
\end{def}

\begin{def}
\lean{MvPoly.sum}
\uses{structure:MvPoly}
The sum of two polynomials in many variables.
\end{def}

\begin{def}
\lean{MvPoly.prod}
\uses{structure:MvPoly}
The product of two polynomials in many variables.
\end{def}

\begin{para}\label{para:BC-distr}
  We shall make frequent use of the Beck-Chevalley isomorphisms and
  of the distributivity law of dependent sums over dependent
  products~\cite{MoerdijkI:weltc}.  Given a cartesian square
  \[
  \xymatrix {
  \cdot \drpullback \ar[r]^g \ar[d]_u & \cdot \ar[d]^v \\
  \cdot \ar[r]_f & \cdot
  }
  \]
  the Beck-Chevalley isomorphisms are
  \[
    \ladj g \, \pbk u \iso \pbk v \, \ladj f \qquad \text{and} \qquad
    \radj g \, \pbk u \iso \pbk v \, \radj f \,.
    \]

  Given maps $C \stackrel u \longrightarrow B \stackrel f \longrightarrow A$,
  we can construct the diagram
  \begin{equation}\label{distr-diag}
  \xymatrixrowsep{40pt}
  \xymatrixcolsep{27pt}
  \vcenter{\hbox{
  \xymatrix @!=0pt {
  &N \drpullback \ar[rr]^g \ar[ld]_e \ar[dd]^{w=\pbk f(v)}&& M \ar[dd]^{v=\radj f (u)} \\
  C \ar[rd]_u && & \\
  &B \ar[rr]_f && A\,,
  }}}
  \end{equation}
  where $w = \pbk f\, \radj f(u)$ and $e$ is the counit
  of $\pbk{f} \adjoint \radj{f}$.
  For such diagrams the following
  distributive law holds:
  \begin{equation}\label{distr-law}
  \radj f \, \ladj u \iso \ladj v \, \radj g \, \pbk e \,.
  \end{equation}


\begin{para}
\label{para:comp}
\lean{MvPoly.comp}
\uses{structure:MvPoly}
We now define the operation of substitution of polynomials, and show that the
  extension of substitution is composition of polynomial functors, as expected.
  In particular, the composite of two polynomial functors is again polynomial.
  Given polynomials
\[
\xymatrix{
 &  B \ar[r]^f  \ar[dl]_{s} & A \ar[dr]^{t} & \\
 I \ar@{}[rrr]|{F} & & & J } \qquad
\xymatrix{
 & D \ar[r]^g \ar[dl]_{u} & C \ar[dr]^{v}   \\
J \ar@{}[rrr]|{G}& & & K }
\]
we say that $F$ is a polynomial from $I$ to $J$ (and $G$ from
$J$ to $K$), and
we define $G\circ F$, the substitution of $F$ into $G$, to be the polynomial
$I \leftarrow N \to M \to K$ constructed via this diagram:
\begin{equation}
\label{equ:compspan}
\xycenter{
  &  &  & N \ar[dl]_{n} \ar[rr]^{p} \ar@{}[dr] |{(iv)}
&  & D'
\ar[dl]^{\varepsilon} \ar[r]^{q} \ar@{}[ddr] |{(ii)} &
M  \ar[dd]^{w} &  \\
  &  & B' \ar[dl]_{m} \ar[rr]^{r} \ar@{} [dr]|{(iii)}
&  & A'  \ar[dr]^{k} \ar[dl]_{h}
\ar@{} [dd] |{(i)}
&  &  &  \\
   & B \ar[rr]^f \ar[dl]_{s} & & A \ar[dr]_{t} &   & D \ar[dl]^{u}
\ar[r]^{g} & C \ar[dr]^{v} &    \\
I  &    & &   & J &   &   & K   }
\end{equation}
Square $(i)$ is cartesian, and $(ii)$ is a distributivity diagram
like \eqref{distr-diag}: $w$ is obtained
by applying $\radj{g}$ to $k$, and $D'$ is the pullback of $M$ along $g$.
The arrow $\varepsilon: D' \to A'$ is the $k$-component of the counit of
the adjunction $\ladj g \adjoint \pbk g$.
Finally, the squares $(iii)$ and $(iv)$ are
cartesian.
\end{para}

\begin{proposition}\label{thm:subst}
\lean{MvPoly.comp.functor}
\uses{structure:MvPoly, def:MvPoly.comp}
There is a natural isomorphism
  $$
  \poly{G\circ F} \iso \poly{G} \circ \poly{F} .
  $$
\end{proposition}

\begin{proof}
  Referring to Diagram~\eqref{equ:compspan} we have
  the following chain of natural isomorphisms:
\begin{eqnarray*}
\extension{G} \circ \extension{F} & = & \ladj{v} \, \radj{g} \, \pbk{u}  \; \ladj{t}  \, \radj{f}
\, \pbk{s} \\
& \iso & \ladj{v} \, \radj{g}  \, \ladj{k}  \, \pbk{h}  \, \radj{f}  \, \pbk{s}\\
 & \iso &
\ladj{v} \, \ladj{w}  \, \radj{q}  \, \pbk{\varepsilon}  \, \pbk{h}  \, \radj{f} \, \pbk{s}\\
& \iso &
\ladj{v} \, \ladj{w}  \, \radj{q}  \, \radj{p}  \, \pbk{n}  \, \pbk{m} \, \pbk{s}\\
& \iso &
\ladj{(v\, w)} \, \radj{(q \, p)}  \, \pbk{(s\, m\, n)}\, \\
& = & \extension{G \circ F}\,.
\end{eqnarray*}
Here we used the Beck-Chevalley isomorphism for the cartesian square
$(i)$, the distributivity law for $(ii)$, Beck-Chevalley isomorphism
for the cartesian squares $(iii)$ and $(iv)$, and finally pseudo-functoriality
of the pullback functors and their adjoints.
\end{proof}
=======
Let $\catC$ be category with pullbacks and terminal object.

\begin{definition}[multivariable polynomial functor]\label{defn:Polynomial}
  \lean{CategoryTheory.MvPoly} \leanok
  A \textbf{polynomial} in $\catC$ from $I$ to $O$ is a triple $(i,p,o)$ where
  $i$, $p$ and $o$ are morphisms in $\catC$ forming the diagram
  $$\polyspan IEBJipo.$$
  The object $I$ is the object of input variables and the object $O$ is the object of output
  variables. The morphism $p$ encodes the arities/exponents.
\end{definition}





\begin{definition}[extension of polynomial functors]
  \label{defn:extension}
  \lean{CategoryTheory.MvPoly.functor} \leanok
  The \textbf{extension} of a polynomial $\polyspan IBAJipo$ is the functor
  $\upP = o_! \, f_* \, i^* \colon \catC / I \to \catC/ O$. Internally, we can define $\upP$ by
  $$\upP \seqbn{X_i}{i \in I} = \seqbn{\sum_{b \in B_j} \prod_{e \in E_b} X_{s(b)}}{j \in J}$$
  A \textbf{polynomial functor} is a functor that is naturally isomorphic to the extension of a polynomial.
  \end{definition}
>>>>>>> refs/remotes/origin/master

\chapter{Test}
<<<<<<< HEAD
\begin{def}
  We define a polynomial over C to be a diagram $F$ in C of shape
  \begin{equation}
  \lean{MvPoly}
  \label{equ:poly}
  \xymatrix{
  I  & B \ar[l]_s \ar[r]^f & A \ar[r]^t & J \, . }
  \end{equation}
  We define $\poly{F}:\slice I \to \slice J$ as the composite
  \[
  \xymatrix{
  \slice{I} \ar[r]^{\pbk{s}} & \slice{B} \ar[r]^{\radj{f}} & \slice{A} \ar[r]^{\ladj{t}} & \slice{J} \, . }
  \]
  We refer to $\poly{F}$ as the polynomial functor associated to $F$, or
  the extension of $F$, and say that $F$ represents $\poly F$.
\end{def}

\begin{def}
\lean{MvPoly.sum}
\uses{structure:MvPoly}
The sum of two polynomials in many variables.
\end{def}

\begin{def}
\lean{MvPoly.prod}
\uses{structure:MvPoly}
The product of two polynomials in many variables.
\end{def}

\begin{para}\label{para:BC-distr}
  We shall make frequent use of the Beck-Chevalley isomorphisms and
  of the distributivity law of dependent sums over dependent
  products~\cite{MoerdijkI:weltc}.  Given a cartesian square
  \[
  \xymatrix {
  \cdot \drpullback \ar[r]^g \ar[d]_u & \cdot \ar[d]^v \\
  \cdot \ar[r]_f & \cdot
  }
  \]
  the Beck-Chevalley isomorphisms are
  \[
    \ladj g \, \pbk u \iso \pbk v \, \ladj f \qquad \text{and} \qquad
    \radj g \, \pbk u \iso \pbk v \, \radj f \,.
    \]

  Given maps $C \stackrel u \longrightarrow B \stackrel f \longrightarrow A$,
  we can construct the diagram
  \begin{equation}\label{distr-diag}
  \xymatrixrowsep{40pt}
  \xymatrixcolsep{27pt}
  \vcenter{\hbox{
  \xymatrix @!=0pt {
  &N \drpullback \ar[rr]^g \ar[ld]_e \ar[dd]^{w=\pbk f(v)}&& M \ar[dd]^{v=\radj f (u)} \\
  C \ar[rd]_u && & \\
  &B \ar[rr]_f && A\,,
  }}}
  \end{equation}
  where $w = \pbk f\, \radj f(u)$ and $e$ is the counit
  of $\pbk{f} \adjoint \radj{f}$.
  For such diagrams the following
  distributive law holds:
  \begin{equation}\label{distr-law}
  \radj f \, \ladj u \iso \ladj v \, \radj g \, \pbk e \,.
  \end{equation}


\begin{para}
\label{para:comp}
\lean{MvPoly.comp}
\uses{structure:MvPoly}
We now define the operation of substitution of polynomials, and show that the
  extension of substitution is composition of polynomial functors, as expected.
  In particular, the composite of two polynomial functors is again polynomial.
  Given polynomials
\[
\xymatrix{
 &  B \ar[r]^f  \ar[dl]_{s} & A \ar[dr]^{t} & \\
 I \ar@{}[rrr]|{F} & & & J } \qquad
\xymatrix{
 & D \ar[r]^g \ar[dl]_{u} & C \ar[dr]^{v}   \\
J \ar@{}[rrr]|{G}& & & K }
\]
we say that $F$ is a polynomial from $I$ to $J$ (and $G$ from
$J$ to $K$), and
we define $G\circ F$, the substitution of $F$ into $G$, to be the polynomial
$I \leftarrow N \to M \to K$ constructed via this diagram:
\begin{equation}
\label{equ:compspan}
\xycenter{
  &  &  & N \ar[dl]_{n} \ar[rr]^{p} \ar@{}[dr] |{(iv)}
&  & D'
\ar[dl]^{\varepsilon} \ar[r]^{q} \ar@{}[ddr] |{(ii)} &
M  \ar[dd]^{w} &  \\
  &  & B' \ar[dl]_{m} \ar[rr]^{r} \ar@{} [dr]|{(iii)}
&  & A'  \ar[dr]^{k} \ar[dl]_{h}
\ar@{} [dd] |{(i)}
&  &  &  \\
   & B \ar[rr]^f \ar[dl]_{s} & & A \ar[dr]_{t} &   & D \ar[dl]^{u}
\ar[r]^{g} & C \ar[dr]^{v} &    \\
I  &    & &   & J &   &   & K   }
\end{equation}
Square $(i)$ is cartesian, and $(ii)$ is a distributivity diagram
like \eqref{distr-diag}: $w$ is obtained
by applying $\radj{g}$ to $k$, and $D'$ is the pullback of $M$ along $g$.
The arrow $\varepsilon: D' \to A'$ is the $k$-component of the counit of
the adjunction $\ladj g \adjoint \pbk g$.
Finally, the squares $(iii)$ and $(iv)$ are
cartesian.
\end{para}

\begin{proposition}\label{thm:subst}
\lean{MvPoly.comp.functor}
\uses{structure:MvPoly, def:MvPoly.comp}
There is a natural isomorphism
  $$
  \poly{G\circ F} \iso \poly{G} \circ \poly{F} .
  $$
\end{proposition}

\begin{proof}
  Referring to Diagram~\eqref{equ:compspan} we have
  the following chain of natural isomorphisms:
\begin{eqnarray*}
\extension{G} \circ \extension{F} & = & \ladj{v} \, \radj{g} \, \pbk{u}  \; \ladj{t}  \, \radj{f}
\, \pbk{s} \\
& \iso & \ladj{v} \, \radj{g}  \, \ladj{k}  \, \pbk{h}  \, \radj{f}  \, \pbk{s}\\
 & \iso &
\ladj{v} \, \ladj{w}  \, \radj{q}  \, \pbk{\varepsilon}  \, \pbk{h}  \, \radj{f} \, \pbk{s}\\
& \iso &
\ladj{v} \, \ladj{w}  \, \radj{q}  \, \radj{p}  \, \pbk{n}  \, \pbk{m} \, \pbk{s}\\
& \iso &
\ladj{(v\, w)} \, \radj{(q \, p)}  \, \pbk{(s\, m\, n)}\, \\
& = & \extension{G \circ F}\,.
\end{eqnarray*}
Here we used the Beck-Chevalley isomorphism for the cartesian square
$(i)$, the distributivity law for $(ii)$, Beck-Chevalley isomorphism
for the cartesian squares $(iii)$ and $(iv)$, and finally pseudo-functoriality
of the pullback functors and their adjoints.
\end{proof}
=======
Let $\catC$ be category with pullbacks and terminal object.

\begin{definition}[multivariable polynomial functor]\label{defn:Polynomial}
  \lean{CategoryTheory.MvPoly} \leanok
  A \textbf{polynomial} in $\catC$ from $I$ to $O$ is a triple $(i,p,o)$ where
  $i$, $p$ and $o$ are morphisms in $\catC$ forming the diagram
  $$\polyspan IEBJipo.$$
  The object $I$ is the object of input variables and the object $O$ is the object of output
  variables. The morphism $p$ encodes the arities/exponents.
\end{definition}





\begin{definition}[extension of polynomial functors]
  \label{defn:extension}
  \lean{CategoryTheory.MvPoly.functor} \leanok
  The \textbf{extension} of a polynomial $\polyspan IBAJipo$ is the functor
  $\upP = o_! \, f_* \, i^* \colon \catC / I \to \catC/ O$. Internally, we can define $\upP$ by
  $$\upP \seqbn{X_i}{i \in I} = \seqbn{\sum_{b \in B_j} \prod_{e \in E_b} X_{s(b)}}{j \in J}$$
  A \textbf{polynomial functor} is a functor that is naturally isomorphic to the extension of a polynomial.
  \end{definition}
>>>>>>> refs/remotes/origin/master


% \bibliography{refs.bib}{}
% \bibliographystyle{alpha}

\end{document}



\title{Polynomial Functors in Lean 4}
\author{Sina Hazratpour}

% This file collects all the blueprint chapters.
% It is not supposed to be built standalone:
% see web.tex and print.tex instead.

\title{Polynomial Functors in Lean4}
\author{Sina Hazratpour}

\begin{document}
\maketitle

\chapter{Locally Cartesian Closed Categories}
\begin{definition}[exponentiable morphism]
  \label{defn:exponentiable-morphism}
  \lean{CategoryTheory.ExponentiableMorphism}
  \leanok
    Suppose $\catC$ is a category with pullbacks.
    A morphism $f \colon A \to B$ in $\catC$ is \textbf{exponentiable} if the pullback functor
    $f^* \colon \catC / B \to \catC / A$ has a right adjoint $f_*$.
    Since $f^*$ always has a left adjoint $f_!$, given by post-composition with $f$, an
    exponentiable morphism $f$ gives rise to an adjoint triple
    \begin{center}\begin{tikzcd}
      & {\catC / B} \\
      \\
      {} & {\catC / A}
      \arrow[""{name=0, anchor=center, inner sep=0}, "{f_*}", bend left, shift left=5, from=1-2, to=3-2]
      \arrow[""{name=1, anchor=center, inner sep=0}, "{f_!}"', bend right, shift right=5, from=1-2, to=3-2]
      \arrow[""{name=2, anchor=center, inner sep=0}, "{f^*}"{description}, from=3-2, to=1-2]
      \arrow["\dashv"{anchor=center}, draw=none, from=1, to=2]
      \arrow["\dashv"{anchor=center}, draw=none, from=2, to=0]
    \end{tikzcd}\end{center}
\end{definition}

\begin{definition}[pushforward functor]
  \label{rmk:pushforward}
  \lean{CategoryTheory.ExponentiableMorphism.pushforward}
  \leanok
  Let $f \colon A \to B$ be an exponentiable morphism in a category $\catC$ with pullbacks.
  We call the right adjoint $f_*$ of the pullback functor $f^*$ the \textbf{pushforward} functor
  along $f$.
\end{definition}

\begin{theorem}[exponentiable morphisms are exponentiable objects of the slices]\label{thm:exponentiable-mor-exponentiable-obj}
  \lean{CategoryTheory.ExponentiableMorphism.OverMkHom}
  \leanok
  \uses{defn:exponentiable-morphism}
  A morphism $f \colon A \to B$ in a category $\catC$ with pullbacks is exponentiable if and only if
  it is an exponentiable object, regarded as an object of the slice $\catC / B$.
\end{theorem}


\begin{definition}[Locally cartesian closed categories]\label{defn:LCCC}
\lean{CategoryTheory.LocallyCartesianClosed} \leanok
  A category with pullbacks is \textbf{locally cartesian closed} if
  is a category $\catC$ with a terminal object $1$ and with all slices $\catC / A$ cartesian closed.
\end{definition}

\chapter{Univaiate Polynomial Functors}
\input{chapter/uvpoly}
\chapter{Multivariate Polynomial Functors}
<<<<<<< HEAD
\begin{def}
  We define a polynomial over C to be a diagram $F$ in C of shape
  \begin{equation}
  \lean{MvPoly}
  \label{equ:poly}
  \xymatrix{
  I  & B \ar[l]_s \ar[r]^f & A \ar[r]^t & J \, . }
  \end{equation}
  We define $\poly{F}:\slice I \to \slice J$ as the composite
  \[
  \xymatrix{
  \slice{I} \ar[r]^{\pbk{s}} & \slice{B} \ar[r]^{\radj{f}} & \slice{A} \ar[r]^{\ladj{t}} & \slice{J} \, . }
  \]
  We refer to $\poly{F}$ as the polynomial functor associated to $F$, or
  the extension of $F$, and say that $F$ represents $\poly F$.
\end{def}

\begin{def}
\lean{MvPoly.sum}
\uses{structure:MvPoly}
The sum of two polynomials in many variables.
\end{def}

\begin{def}
\lean{MvPoly.prod}
\uses{structure:MvPoly}
The product of two polynomials in many variables.
\end{def}

\begin{para}\label{para:BC-distr}
  We shall make frequent use of the Beck-Chevalley isomorphisms and
  of the distributivity law of dependent sums over dependent
  products~\cite{MoerdijkI:weltc}.  Given a cartesian square
  \[
  \xymatrix {
  \cdot \drpullback \ar[r]^g \ar[d]_u & \cdot \ar[d]^v \\
  \cdot \ar[r]_f & \cdot
  }
  \]
  the Beck-Chevalley isomorphisms are
  \[
    \ladj g \, \pbk u \iso \pbk v \, \ladj f \qquad \text{and} \qquad
    \radj g \, \pbk u \iso \pbk v \, \radj f \,.
    \]

  Given maps $C \stackrel u \longrightarrow B \stackrel f \longrightarrow A$,
  we can construct the diagram
  \begin{equation}\label{distr-diag}
  \xymatrixrowsep{40pt}
  \xymatrixcolsep{27pt}
  \vcenter{\hbox{
  \xymatrix @!=0pt {
  &N \drpullback \ar[rr]^g \ar[ld]_e \ar[dd]^{w=\pbk f(v)}&& M \ar[dd]^{v=\radj f (u)} \\
  C \ar[rd]_u && & \\
  &B \ar[rr]_f && A\,,
  }}}
  \end{equation}
  where $w = \pbk f\, \radj f(u)$ and $e$ is the counit
  of $\pbk{f} \adjoint \radj{f}$.
  For such diagrams the following
  distributive law holds:
  \begin{equation}\label{distr-law}
  \radj f \, \ladj u \iso \ladj v \, \radj g \, \pbk e \,.
  \end{equation}


\begin{para}
\label{para:comp}
\lean{MvPoly.comp}
\uses{structure:MvPoly}
We now define the operation of substitution of polynomials, and show that the
  extension of substitution is composition of polynomial functors, as expected.
  In particular, the composite of two polynomial functors is again polynomial.
  Given polynomials
\[
\xymatrix{
 &  B \ar[r]^f  \ar[dl]_{s} & A \ar[dr]^{t} & \\
 I \ar@{}[rrr]|{F} & & & J } \qquad
\xymatrix{
 & D \ar[r]^g \ar[dl]_{u} & C \ar[dr]^{v}   \\
J \ar@{}[rrr]|{G}& & & K }
\]
we say that $F$ is a polynomial from $I$ to $J$ (and $G$ from
$J$ to $K$), and
we define $G\circ F$, the substitution of $F$ into $G$, to be the polynomial
$I \leftarrow N \to M \to K$ constructed via this diagram:
\begin{equation}
\label{equ:compspan}
\xycenter{
  &  &  & N \ar[dl]_{n} \ar[rr]^{p} \ar@{}[dr] |{(iv)}
&  & D'
\ar[dl]^{\varepsilon} \ar[r]^{q} \ar@{}[ddr] |{(ii)} &
M  \ar[dd]^{w} &  \\
  &  & B' \ar[dl]_{m} \ar[rr]^{r} \ar@{} [dr]|{(iii)}
&  & A'  \ar[dr]^{k} \ar[dl]_{h}
\ar@{} [dd] |{(i)}
&  &  &  \\
   & B \ar[rr]^f \ar[dl]_{s} & & A \ar[dr]_{t} &   & D \ar[dl]^{u}
\ar[r]^{g} & C \ar[dr]^{v} &    \\
I  &    & &   & J &   &   & K   }
\end{equation}
Square $(i)$ is cartesian, and $(ii)$ is a distributivity diagram
like \eqref{distr-diag}: $w$ is obtained
by applying $\radj{g}$ to $k$, and $D'$ is the pullback of $M$ along $g$.
The arrow $\varepsilon: D' \to A'$ is the $k$-component of the counit of
the adjunction $\ladj g \adjoint \pbk g$.
Finally, the squares $(iii)$ and $(iv)$ are
cartesian.
\end{para}

\begin{proposition}\label{thm:subst}
\lean{MvPoly.comp.functor}
\uses{structure:MvPoly, def:MvPoly.comp}
There is a natural isomorphism
  $$
  \poly{G\circ F} \iso \poly{G} \circ \poly{F} .
  $$
\end{proposition}

\begin{proof}
  Referring to Diagram~\eqref{equ:compspan} we have
  the following chain of natural isomorphisms:
\begin{eqnarray*}
\extension{G} \circ \extension{F} & = & \ladj{v} \, \radj{g} \, \pbk{u}  \; \ladj{t}  \, \radj{f}
\, \pbk{s} \\
& \iso & \ladj{v} \, \radj{g}  \, \ladj{k}  \, \pbk{h}  \, \radj{f}  \, \pbk{s}\\
 & \iso &
\ladj{v} \, \ladj{w}  \, \radj{q}  \, \pbk{\varepsilon}  \, \pbk{h}  \, \radj{f} \, \pbk{s}\\
& \iso &
\ladj{v} \, \ladj{w}  \, \radj{q}  \, \radj{p}  \, \pbk{n}  \, \pbk{m} \, \pbk{s}\\
& \iso &
\ladj{(v\, w)} \, \radj{(q \, p)}  \, \pbk{(s\, m\, n)}\, \\
& = & \extension{G \circ F}\,.
\end{eqnarray*}
Here we used the Beck-Chevalley isomorphism for the cartesian square
$(i)$, the distributivity law for $(ii)$, Beck-Chevalley isomorphism
for the cartesian squares $(iii)$ and $(iv)$, and finally pseudo-functoriality
of the pullback functors and their adjoints.
\end{proof}
=======
Let $\catC$ be category with pullbacks and terminal object.

\begin{definition}[multivariable polynomial functor]\label{defn:Polynomial}
  \lean{CategoryTheory.MvPoly} \leanok
  A \textbf{polynomial} in $\catC$ from $I$ to $O$ is a triple $(i,p,o)$ where
  $i$, $p$ and $o$ are morphisms in $\catC$ forming the diagram
  $$\polyspan IEBJipo.$$
  The object $I$ is the object of input variables and the object $O$ is the object of output
  variables. The morphism $p$ encodes the arities/exponents.
\end{definition}





\begin{definition}[extension of polynomial functors]
  \label{defn:extension}
  \lean{CategoryTheory.MvPoly.functor} \leanok
  The \textbf{extension} of a polynomial $\polyspan IBAJipo$ is the functor
  $\upP = o_! \, f_* \, i^* \colon \catC / I \to \catC/ O$. Internally, we can define $\upP$ by
  $$\upP \seqbn{X_i}{i \in I} = \seqbn{\sum_{b \in B_j} \prod_{e \in E_b} X_{s(b)}}{j \in J}$$
  A \textbf{polynomial functor} is a functor that is naturally isomorphic to the extension of a polynomial.
  \end{definition}
>>>>>>> refs/remotes/origin/master

\chapter{Test}
<<<<<<< HEAD
\begin{def}
  We define a polynomial over C to be a diagram $F$ in C of shape
  \begin{equation}
  \lean{MvPoly}
  \label{equ:poly}
  \xymatrix{
  I  & B \ar[l]_s \ar[r]^f & A \ar[r]^t & J \, . }
  \end{equation}
  We define $\poly{F}:\slice I \to \slice J$ as the composite
  \[
  \xymatrix{
  \slice{I} \ar[r]^{\pbk{s}} & \slice{B} \ar[r]^{\radj{f}} & \slice{A} \ar[r]^{\ladj{t}} & \slice{J} \, . }
  \]
  We refer to $\poly{F}$ as the polynomial functor associated to $F$, or
  the extension of $F$, and say that $F$ represents $\poly F$.
\end{def}

\begin{def}
\lean{MvPoly.sum}
\uses{structure:MvPoly}
The sum of two polynomials in many variables.
\end{def}

\begin{def}
\lean{MvPoly.prod}
\uses{structure:MvPoly}
The product of two polynomials in many variables.
\end{def}

\begin{para}\label{para:BC-distr}
  We shall make frequent use of the Beck-Chevalley isomorphisms and
  of the distributivity law of dependent sums over dependent
  products~\cite{MoerdijkI:weltc}.  Given a cartesian square
  \[
  \xymatrix {
  \cdot \drpullback \ar[r]^g \ar[d]_u & \cdot \ar[d]^v \\
  \cdot \ar[r]_f & \cdot
  }
  \]
  the Beck-Chevalley isomorphisms are
  \[
    \ladj g \, \pbk u \iso \pbk v \, \ladj f \qquad \text{and} \qquad
    \radj g \, \pbk u \iso \pbk v \, \radj f \,.
    \]

  Given maps $C \stackrel u \longrightarrow B \stackrel f \longrightarrow A$,
  we can construct the diagram
  \begin{equation}\label{distr-diag}
  \xymatrixrowsep{40pt}
  \xymatrixcolsep{27pt}
  \vcenter{\hbox{
  \xymatrix @!=0pt {
  &N \drpullback \ar[rr]^g \ar[ld]_e \ar[dd]^{w=\pbk f(v)}&& M \ar[dd]^{v=\radj f (u)} \\
  C \ar[rd]_u && & \\
  &B \ar[rr]_f && A\,,
  }}}
  \end{equation}
  where $w = \pbk f\, \radj f(u)$ and $e$ is the counit
  of $\pbk{f} \adjoint \radj{f}$.
  For such diagrams the following
  distributive law holds:
  \begin{equation}\label{distr-law}
  \radj f \, \ladj u \iso \ladj v \, \radj g \, \pbk e \,.
  \end{equation}


\begin{para}
\label{para:comp}
\lean{MvPoly.comp}
\uses{structure:MvPoly}
We now define the operation of substitution of polynomials, and show that the
  extension of substitution is composition of polynomial functors, as expected.
  In particular, the composite of two polynomial functors is again polynomial.
  Given polynomials
\[
\xymatrix{
 &  B \ar[r]^f  \ar[dl]_{s} & A \ar[dr]^{t} & \\
 I \ar@{}[rrr]|{F} & & & J } \qquad
\xymatrix{
 & D \ar[r]^g \ar[dl]_{u} & C \ar[dr]^{v}   \\
J \ar@{}[rrr]|{G}& & & K }
\]
we say that $F$ is a polynomial from $I$ to $J$ (and $G$ from
$J$ to $K$), and
we define $G\circ F$, the substitution of $F$ into $G$, to be the polynomial
$I \leftarrow N \to M \to K$ constructed via this diagram:
\begin{equation}
\label{equ:compspan}
\xycenter{
  &  &  & N \ar[dl]_{n} \ar[rr]^{p} \ar@{}[dr] |{(iv)}
&  & D'
\ar[dl]^{\varepsilon} \ar[r]^{q} \ar@{}[ddr] |{(ii)} &
M  \ar[dd]^{w} &  \\
  &  & B' \ar[dl]_{m} \ar[rr]^{r} \ar@{} [dr]|{(iii)}
&  & A'  \ar[dr]^{k} \ar[dl]_{h}
\ar@{} [dd] |{(i)}
&  &  &  \\
   & B \ar[rr]^f \ar[dl]_{s} & & A \ar[dr]_{t} &   & D \ar[dl]^{u}
\ar[r]^{g} & C \ar[dr]^{v} &    \\
I  &    & &   & J &   &   & K   }
\end{equation}
Square $(i)$ is cartesian, and $(ii)$ is a distributivity diagram
like \eqref{distr-diag}: $w$ is obtained
by applying $\radj{g}$ to $k$, and $D'$ is the pullback of $M$ along $g$.
The arrow $\varepsilon: D' \to A'$ is the $k$-component of the counit of
the adjunction $\ladj g \adjoint \pbk g$.
Finally, the squares $(iii)$ and $(iv)$ are
cartesian.
\end{para}

\begin{proposition}\label{thm:subst}
\lean{MvPoly.comp.functor}
\uses{structure:MvPoly, def:MvPoly.comp}
There is a natural isomorphism
  $$
  \poly{G\circ F} \iso \poly{G} \circ \poly{F} .
  $$
\end{proposition}

\begin{proof}
  Referring to Diagram~\eqref{equ:compspan} we have
  the following chain of natural isomorphisms:
\begin{eqnarray*}
\extension{G} \circ \extension{F} & = & \ladj{v} \, \radj{g} \, \pbk{u}  \; \ladj{t}  \, \radj{f}
\, \pbk{s} \\
& \iso & \ladj{v} \, \radj{g}  \, \ladj{k}  \, \pbk{h}  \, \radj{f}  \, \pbk{s}\\
 & \iso &
\ladj{v} \, \ladj{w}  \, \radj{q}  \, \pbk{\varepsilon}  \, \pbk{h}  \, \radj{f} \, \pbk{s}\\
& \iso &
\ladj{v} \, \ladj{w}  \, \radj{q}  \, \radj{p}  \, \pbk{n}  \, \pbk{m} \, \pbk{s}\\
& \iso &
\ladj{(v\, w)} \, \radj{(q \, p)}  \, \pbk{(s\, m\, n)}\, \\
& = & \extension{G \circ F}\,.
\end{eqnarray*}
Here we used the Beck-Chevalley isomorphism for the cartesian square
$(i)$, the distributivity law for $(ii)$, Beck-Chevalley isomorphism
for the cartesian squares $(iii)$ and $(iv)$, and finally pseudo-functoriality
of the pullback functors and their adjoints.
\end{proof}
=======
Let $\catC$ be category with pullbacks and terminal object.

\begin{definition}[multivariable polynomial functor]\label{defn:Polynomial}
  \lean{CategoryTheory.MvPoly} \leanok
  A \textbf{polynomial} in $\catC$ from $I$ to $O$ is a triple $(i,p,o)$ where
  $i$, $p$ and $o$ are morphisms in $\catC$ forming the diagram
  $$\polyspan IEBJipo.$$
  The object $I$ is the object of input variables and the object $O$ is the object of output
  variables. The morphism $p$ encodes the arities/exponents.
\end{definition}





\begin{definition}[extension of polynomial functors]
  \label{defn:extension}
  \lean{CategoryTheory.MvPoly.functor} \leanok
  The \textbf{extension} of a polynomial $\polyspan IBAJipo$ is the functor
  $\upP = o_! \, f_* \, i^* \colon \catC / I \to \catC/ O$. Internally, we can define $\upP$ by
  $$\upP \seqbn{X_i}{i \in I} = \seqbn{\sum_{b \in B_j} \prod_{e \in E_b} X_{s(b)}}{j \in J}$$
  A \textbf{polynomial functor} is a functor that is naturally isomorphic to the extension of a polynomial.
  \end{definition}
>>>>>>> refs/remotes/origin/master


% \bibliography{refs.bib}{}
% \bibliographystyle{alpha}

\end{document}

